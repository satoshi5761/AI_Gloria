\documentclass{article}
\usepackage[a4paper, margin=1cm]{geometry}
\usepackage{tabularx}
\usepackage{graphicx}
\usepackage{wrapfig}
\usepackage{amsmath}


\title{\textbf{LAPORAN TUGAS}}
\author{Sistem Evakuasi Kebakaran}
\date{}

\begin{document}

\maketitle
\begin{center}
    \includegraphics[scale=0.4]{logo-ukdw.png}
\end{center}

\begin{table}[h]
    \centering
    \renewcommand{\arraystretch}{1.5}
    \begin{tabularx}{\textwidth}{|c|X|}
        \hline
        Mata kuliah	& TI0263 – Kecerdasan Buatan (GRUP C) \\
        \hline
        Dosen Pengampu & Gloria Virginia, S.Kom., MAI., Ph.D \\
        \hline
        Nama Kelompok & C8 \\
        \hline 
        Anggota Kelompok & 
        \begin{minipage}{\textwidth}
            \vspace{5px}
            \begin{enumerate}
                \item 71230985 - Tomas Becket
                \item 71231002 – Philip Deric Kho  
                \item 71231015 - Karel Marley Bala Bakior
                \item 71231017 - Paulus Ungirwalu
                \item 71231061 - Syendhi Reswara. S
            \end{enumerate}
            \vspace{5px}
        \end{minipage} \\
        \hline
        Deklarasi & Dengan ini kami menyatakan bahwa tugas ini merupakan hasil karya kelompok kami, tidak ada manipulasi data serta bukan merupakan plagiasi dari karya orang lain. \\
        \hline
    \end{tabularx}
\end{table}


\vfill

\noindent
\makebox[0pt][l]{%
  \begin{tabular}{@{}c@{}}
  \includegraphics[scale=0.2]{fti-ukdw.png}
  \end{tabular}%
}\hfill
\textbf{\begin{tabular}{@{}c@{}}
    \textbf{UNIVERSITAS KRISTEN DUTA WACANA} \\
Fakultas Teknologi Informasi \\ 
Program Studi Informatika \\ 
  \end{tabular}%
}\hfill
\makebox[0pt][r]{%
  \begin{tabular}{@{}c@{}}
  \includegraphics[scale=0.133]{tiukdw.jpg}
  \end{tabular}%
}

\section*{Tugas2}
\subsection{Deskripsi}
Bang Jeffry adalah seorang mahasiswa FTI UKDW yang ingin melakukan registrasi
matakuliah. Bang Jeffry sangat menyukai matakuliah yang berhubungan dengan AI
dan game. Bang Jeffry sangat tidak menyukai jam matakuliah yang terlalu pagi atau 
terlalu malam. Dikarenakan bang Jeffry harus mengambil full SKS, maka dia akan mengambil
matakuliah yang mudah untuk menambah SKS-nya. Ada beberapa kriteria matkul yang menjadi pilihan Bang Jeffry:
\begin{itemize}
  \item Matkul yang sedang hype
  \item Matkul yang tidak ada coding
  \item Matkul yang diampu oleh dosen yang murah hati
  \item Matkul yang diluar IT (matkul bebas; Bahasa Inggris, Pendidikan Perdamaian)
\end{itemize}


Bang Jeffry juga sangat memilih-milih dosen. Terdapat 2 dosen yang dia sukai yaitu 
Bu Gloria dan Pak Karel. Selain itu, dia juga sangat menyukai dosen yang tidak mudah mengamuk dan
murah nilai atau yang biasa disebut dengan dosen \emph{chill}. Jika tidak ada dosen yang chill, maka
Bang Jeffry akan mengambil matkul tersebut di tahun berikutnya.


Dikarenakan Bang Jeffry sangat pemalu, dia akan memilih matkul jika teman-temannya 
juga memilih matkul itu.


\indent Bang Jeffry akan mengulang matakuliah jika dan hanya jika matakuliah tersebut bernilai E.

\subsection{Production System}
\begin{enumerate}
    \item \textbf{If} mata kuliah berhubungan denga AI atau Game Development, \textbf{then} tandai sebagai "diminati".
    \item \textbf{If} jadwal mata kuliah terlalu pagi atau terlalu malam, \textbf{then} tandai sebagai "tidak diminati".
    \item \textbf{If} mata kuliah tidak memiliki coding, sedang tren, merupakan mata kuliah di luar IT, atau diajarkan oleh dosen yang murah hati, \textbf{then} tandai sebagai "mudah".
    \item \textbf{If} total SKS yang diambil $<$ 24, \textbf{then} pilih mata kuliah dari kategori "mudah" terlebih dahulu.
    \item \textbf{If} mata kuliah diajarkan oleh Bu Gloria atau Pak Karel, \textbf{then} pilih mata kuliah tersebut secepatnya.
    \item \textbf{If} dosen tidak \emph{chill} (mudah marah atau pelit dalam memberi nilai), \textbf{then} tunda pengambilan mata kuliah ke tahun berikutnya.
    \item \textbf{If} ada setidaknya satu teman yang mengambil mata kuliah tersebut, \textbf{then} tandai sebagai "diminati".
    \item \textbf{If} mata kuliah sebelumnya mendapat nilai E, \textbf{then} mata kuliah tersebut harus diulang.
    \item \textbf{If} masih ada sisa SKS yang perlu diambil dan tidak ada kriteria lain yang sesuai, \textbf{then} pilih mata kuliah secara acak.
\end{enumerate}


\vfill

\noindent
\makebox[0pt][l]{%
  \begin{tabular}{@{}c@{}}
  \includegraphics[scale=0.2]{fti-ukdw.png}
  \end{tabular}%
}\hfill
\textbf{\begin{tabular}{@{}c@{}}
    \textbf{UNIVERSITAS KRISTEN DUTA WACANA} \\
Fakultas Teknologi Informasi \\ 
Program Studi Informatika \\ 
  \end{tabular}%
}\hfill
\makebox[0pt][r]{%
  \begin{tabular}{@{}c@{}}
  \includegraphics[scale=0.133]{tiukdw.jpg}
  \end{tabular}%
}
\end{document}