\documentclass{article}
\usepackage[a4paper, margin=1cm]{geometry}
\usepackage{tabularx}
\usepackage{graphicx}
\usepackage{wrapfig}
\usepackage{amsmath}


\title{\textbf{LAPORAN TUGAS}}
\author{Sistem Evakuasi Kebakaran}
\date{}

\begin{document}

\maketitle
\begin{center}
    \includegraphics[scale=0.4]{logo-ukdw.png}
\end{center}

\begin{table}[h]
    \centering
    \renewcommand{\arraystretch}{1.5}
    \begin{tabularx}{\textwidth}{|c|X|}
        \hline
        Mata kuliah	& TI0263 – Kecerdasan Buatan (GRUP C) \\
        \hline
        Dosen Pengampu & Gloria Virginia, S.Kom., MAI., Ph.D \\
        \hline
        Nama Kelompok & C8 \\
        \hline 
        Anggota Kelompok & 
        \begin{minipage}{\textwidth}
            \vspace{5px}
            \begin{enumerate}
                \item 71230985 - Tomas Becket
                \item 71231002 – Philip Deric Kho  
                \item 71231015 - Karel Marley Bala Bakior
                \item 71231017 - Paulus Ungirwalu
                \item 71231061 - Syendhi Reswara. S
            \end{enumerate}
            \vspace{5px}
        \end{minipage} \\
        \hline
        Deklarasi & Dengan ini kami menyatakan bahwa tugas ini merupakan hasil karya kelompok kami, tidak ada manipulasi data serta bukan merupakan plagiasi dari karya orang lain. \\
        \hline
    \end{tabularx}
\end{table}

\section*{Tugas1}
\subsection{Topik}
Weighted Graph
\subsection{Hasil Akhir}
Kasus: terjadi kebakaran pada suatu tempat (kami belum memastikan 
tempat apa yang akan kami pakai), Anda sebagai salah satu orang 
yang terjebak dalam kebakaran tersebut ingin mengetahui jalur 
evakuasi mana yang tercepat dan teraman untuk dilalui demi keselamatan
diri Anda. Program kami akan membantu Anda untuk mencari jalur tersebut.
Sistem dapat menerima input lokasi pengguna dan memiliki pengetahuan
akan titik-titik yang mengalami kebakaran dan yang tidak mengalami kebakaran.
Sistem akan mencoba mencari jalur untuk pengguna berdasarkan pengetahuan tersebut
menggunakan inferensi yang tepat. Jika memungkinkan, pengguna dapat melihat 
jalur evakuasi yang diberikan oleh program secara visual.
Perlu diketahui bahwa deskripsi ini dapat berubah suatu waktu.


\vfill

\noindent
\makebox[0pt][l]{%
  \begin{tabular}{@{}c@{}}
  \includegraphics[scale=0.2]{fti-ukdw.png}
  \end{tabular}%
}\hfill
\textbf{\begin{tabular}{@{}c@{}}
    \textbf{UNIVERSITAS KRISTEN DUTA WACANA} \\
Fakultas Teknologi Informasi \\ 
Program Studi Informatika \\ 
  \end{tabular}%
}\hfill
\makebox[0pt][r]{%
  \begin{tabular}{@{}c@{}}
  \includegraphics[scale=0.133]{tiukdw.jpg}
  \end{tabular}%
}

\subsection{Knowledge Based System}
- Sistem mempunyai pemahaman struktur dan tata letak gedung. \\
Sistem memiliki data denah 13 gedung di UKDW dan lokasi pintu keluar \\
- Sistem memiliki algoritma pencarian jalur terbaik seperti menggunakan Dijkstra atau A* untuk menentukan rute teraman

\subsection{Pembagian Tugas}
\emph{Pembagian Tugas ini masih bersifat general
dan belum eksplisit untuk} \boxed{saat \; ini}
\begin{itemize}
  \item Tomas: AI
  \item Philip: AI
  \item Karel: AI
  \item Paulus: UI
  \item Syendhi: UI
\end{itemize}

\vfill

\noindent
\makebox[0pt][l]{%
  \begin{tabular}{@{}c@{}}
  \includegraphics[scale=0.2]{fti-ukdw.png}
  \end{tabular}%
}\hfill
\textbf{\begin{tabular}{@{}c@{}}
    \textbf{UNIVERSITAS KRISTEN DUTA WACANA} \\
Fakultas Teknologi Informasi \\ 
Program Studi Informatika \\ 
  \end{tabular}%
}\hfill
\makebox[0pt][r]{%
  \begin{tabular}{@{}c@{}}
  \includegraphics[scale=0.133]{tiukdw.jpg}
  \end{tabular}%
}
\end{document}